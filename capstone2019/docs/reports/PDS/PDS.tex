%%%%%%%%%%%%%%%%%%%%%%%%%%%%%%%%%%%%%%%%%
% Academic Title Page
% LaTeX Template
% Version 2.0 (17/7/17)
%
% This template was downloaded from:
% http://www.LaTeXTemplates.com
%
% Original author:
% WikiBooks (LaTeX - Title Creation) with modifications by:
% Vel (vel@latextemplates.com)
%
% License:
% CC BY-NC-SA 3.0 (http://creativecommons.org/licenses/by-nc-sa/3.0/)
% 
% Instructions for using this template:
% This title page is capable of being compiled as is. This is not useful for 
% including it in another document. To do this, you have two options: 
%
% 1) Copy/paste everything between \begin{document} and \end{document} 
% starting at \begin{titlepage} and paste this into another LaTeX file where you 
% want your title page.
% OR
% 2) Remove everything outside the \begin{titlepage} and \end{titlepage}, rename
% this file and move it to the same directory as the LaTeX file you wish to add it to. 
% Then add \input{./<new filename>.tex} to your LaTeX file where you want your
% title page.
%
%%%%%%%%%%%%%%%%%%%%%%%%%%%%%%%%%%%%%%%%%

%----------------------------------------------------------------------------------------
%	PACKAGES AND OTHER DOCUMENT CONFIGURATIONS
%----------------------------------------------------------------------------------------

\documentclass[11pt]{article}

\usepackage[utf8]{inputenc} % Required for inputting international characters
\usepackage[T1]{fontenc} % Output font encoding for international characters
\usepackage{graphics}
\usepackage{graphicx}
\usepackage{indentfirst}
\graphicspath{{images/} }
\usepackage{mathpazo} % Palatino font

\begin{document}

%----------------------------------------------------------------------------------------
%	TITLE PAGE
%----------------------------------------------------------------------------------------

\begin{titlepage} % Suppresses displaying the page number on the title page and the subsequent page counts as page 1
	\newcommand{\HRule}{\rule{\linewidth}{0.5mm}} % Defines a new command for horizontal lines, change thickness here
	\centering % Centre everything on the page
	
	%------------------------------------------------
	%	Headings
	%------------------------------------------------
	
	\textsc{Portland State University}\\[1.5cm] % Main heading such as the name of your university/college
	
	\textsc{ECE 412}\\[0.5cm] % Major heading such as course name
	
	\textsc{Capstone Project 2020}\\[0.5cm] % Minor heading such as course title
	
	%------------------------------------------------
	%	Title
	%------------------------------------------------
	
	\HRule\\[0.4cm]
	
	{\huge\bfseries RISC-V Cleanflight Autopilot}\\[0.4cm] % Title of your document
	
	\HRule\\[1.5cm]
	
	%------------------------------------------------
	%	Author(s)
	%------------------------------------------------
	
	\begin{minipage}{\textwidth}
		\begin{center}
			Bliss Brass \normalfont{- brass@pdx.edu}\\ % Your name
		\end{center}
	\end{minipage}

	\begin{minipage}{\textwidth}
		\begin{center}
			Eric Schulte \normalfont{- eschulte@pdx.edu}\\ % Your name
		\end{center}
	\end{minipage}

	\begin{minipage}{\textwidth}
		\begin{center}
			Nikolay Nikolov \normalfont{- nnikolov@pdx.edu}\\ % Your name
		\end{center}
	\end{minipage}

	\begin{minipage}{\textwidth}
		\begin{center}
			Ruben Maldonado \normalfont{- mruben@pdx.edu}\\ % Your name
		\end{center}
	\end{minipage}
\newline
\newline

	\begin{minipage}{\textwidth}
		\begin{center}
			\Large
			\textit{Advisor:}\\ Roy Kravitz % Supervisor's name
		\end{center}
	\end{minipage}
\newline
\newline

	\begin{minipage}{\textwidth}
		\begin{center}
			\Large
			\textit{Sponsor:}\\ Galois % Supervisor's name
		\end{center}
	\end{minipage}
	%------------------------------------------------
	%	Date
	%------------------------------------------------
	
	\vfill\vfill\vfill % Position the date 3/4 down the remaining page
	{\large\today} % Date, change the \today to a set date if you want to be precise
	
	%------------------------------------------------
	%	Logo
	%------------------------------------------------
	
	\vfill\vfill
	%----------------------------------------------------------------------------------------
	\vfill % Push the date up 1/4 of the remaining page
	
\end{titlepage}
\pagestyle{plain}
\tableofcontents
\newpage

\section{Project Summary}

The project’s main objective is porting the open-source flight controller software Cleanflight to the HiFive1 Revision B RISC-V development board and building a racing drone around it as a demonstrator. 

The team will identify the critical components required for building and operating a racing drone using the Cleanflight flight controller and modify the existing drivers and IMU in the repository to accommodate RISC-V architecture.

At the successful completion of the project the team will deliver to the customer(sponsor of the capstone) a fully integrated drone that uses the open-source flight controller and executes accurately fundamental for the operation of the drone commands. Examples of the drone’s behavior after successful porting are the user pushes up/down on the controller and the drones increases/decreases its height in a vertical fashion and user pushes left/right on the controller and the drone changes  direction in regards of the controller’s input.

\section{Purpose and Need}

The purpose of the project is to demonstrate a practical application using RISC-V micro-controller. RISC-V is an open source Instruction Set Architecture. In the recent years companies such as Apple, Facebook, Google and Samsung have invested resources in creating their own silicon. However, investing into an entirely new chip architecture is expensive. RISC-V is a free and open ISA, it enabling companies to innovate and build their own silicon. Moreover, RISC-V micro controllers can be cheap,fast,low power and are versatile accommodating a wide variety of applications . 

There is a need for more practical applications using RISC-V micro controllers. The experience we are gathering and documenting through completing successfully porting the Cleanflight Controller will assist our customer to diversify its current applications using RISC-V ISA.

\pagebreak

\section{Business Drivers , Significance and Brief Market Analysis}

The successful completion of our project is going to demonstrate that practical applications for RISC-V are attainable, and porting already existing drivers for Arm Micro-controllers is a viable and well documented process. Therefore, in the future our customer can implement it’s own silicon and drivers in house and tailor its products to meet the market demands. 

For us the successful completion of the project translates to a useful experience with the rapidly upcoming Instruction Set Architecture. More, we will gain experience with RISC-V ,further our knowledge related to device drivers , working on a large open-source project, and how to construct a Master Test Plan that includes Unit Test Plan, Integration Test Plan, System Test Plan, and Acceptance Test Plan.

\section{Benefits and Costs}


\section{Implementation Method}

\section{Timeline}

\section{Milestones}

\section{Requirements}

\section{Risks}

\section{Approval}



%----------------------------------------------------------------------------------------

\end{document}
